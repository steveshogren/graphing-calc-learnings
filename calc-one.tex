\documentclass{article}

\begin{document}

Hello world


\begin{equation}
k_{n+1} = n^2 + k_n^2 - k_{n-1}
\end{equation}

Khan Academy Question 2
Simplify the following for h
\begin{equation}
S=2\pi r(r+h)
\end{equation}
\begin{equation}
S/2\pi r=r+h
\end{equation}
\begin{equation}
h=(S/2\pi r)-r
\end{equation}

\begin{equation}
  S=250;
  r=4;
  h=(S/2\pi r)-r;
  h= 5.94718394325
\end{equation}


Does the following have an x-intercept?

\begin{equation}
  f(x) = x(x+4)
\end{equation}

% To graph, run calc. Type your range: e.g. _3 .. 3 then type the desired
% formula: x(x+4) and gf to run the graphing calc. It crosses the middle line, so
% I will guess yes, it has an x-intercept.

What is the rough y-intercept of $$[-3..3] f(x) = (x+3)(1-x) => ~2.8$$

The y-intercept is the value at f(0)
The x-intercept is the value where f(x) = 0
Periodic is a repeating or looping graph
An even graph has f(x) = f(-x) or will look the same reflected over the y-axis
An odd graph has -f(x) = f(-x) or will look mirrored over the x axis

What is the approximate maximum value of the formula:

\begin{equation}
  max(-x^2 + 6x - 1)  = ~8
\end{equation}


\section{New section}
Solve for h
\begin{equation}
  A= 1/​2 ​​ (b+c)h
\end{equation}
\begin{equation}
  A/(1/2(b+c))= h
\end{equation}
\begin{equation}
  2A/(b+c)= h
\end{equation}

To store a variable and solve, use ``st'' and ``=''.

For example: I want to solve $$E/c^2$$
I type it exactly, then type my value for E, type stE, type my value for c, and type stc

\section{New section}
What is the height of a trapezoid with one base equal to 20 m, the other base equal to 7m, and an area of 135m? It is 10m.

Solve $$m=E/(c^2)$$ for $$c = 300,000,000 m/s$$ and $$E = 1.8e14J$$

$$m = 2e-3$$

\section{New section}
The Area of a trapezoid is $$A=(1/2)(b+c)h $$ solve for b
$$A/(1/2)(h)=b+c$$
$$(A/(1/2)h)-c=b$$
$$b = 2 A / h - c $$

When $$A=80ft^2$$ $$h=10ft$$ $$c=5ft$$

$$b=11$$

\section{New section}
Kinetic Energy object in motion $$K=1/2mv^2$$ solve for velocity v
$$m=800$$
$$K=100000$$
$$sqrt(2K/m)$$
$$sqrt(2 100000/800) = 15.8113883008$$

\section{New section}
Net income formula is $$NI=(SP-VC)(V)-FC$$ solve for V where
$$NI=5000$$
$$SP=40$$
$$VC=15$$
$$FC=1000$$
$$NI+FC=(SP-VC)V$$
$$V=(NI+FC)/(SP-VC)$$
$$V=(NI+FC)/(SP-VC)$$
$$=240$$

\section{New section}

What is the y-intercept or x(0) of $$-(x-1)(x+3)$$ it should be $$~3$$

\section{New section}

What is the limit as x approaches 0? $$\lim_{x\to0} \frac{cos2x - cos3x}{x^2}$$

Appears to be 2.5
3.99631755994

\section{Factoring}

Use the quadratic formula:
$$(x+a)(x+b)$$
$$x^2+xb+ax+ab$$
$$x^2+(a+b)x+ab$$

To factor $$x^2+10x+9$$
$$a+b = 10$$
$$ab = 9$$
Factors of 9 are 1,3,9
$$1+9 = 10$$
$$1*9 = 9$$
So
$$(x+1)(x+9)$$

To factor $$x^2-1$$

Use Sum of cubes:
$$a^3+b^3$$
$$(a+b)(a^2-ab+b^2)$$
Use Difference of cubes:
$$a^3-b^3$$
$$(a-b)(a^2+ab+b^2)$$

To solve a division:

$$\frac{x^3-8}{x-2}$$

Power Rule:
$$f(x)=x^n$$
$$f'(x)=nx^{n-1}$$


\section{Exponents}

Add to combine when the terms multiply:

$$x^3*x^4$$
$$x^{3+4}$$
$$x^7$$

Multiply when terms ``compound'''

$$(x^3)^4$$
$$x^{3*4}$$
$$x^{12}$$


You can separate the terms:
$$(ab)^3 = a^3b^3$$

Negative exponents are the same as a fractional positive number:

$$a^{-3} = (1/a^3)$$


$$(4^{-3}*2^{-3})^0$$
$$(8^{-3})^0$$
$$8^{0}$$
$$1$$

To find the solutions of x:
$$5x^2+15x-50=0$$
$$5x^2+15x=50$$
$$5(x^2+3x)=50$$
$$x^2+3x=10$$
$$x^2+3x-10=0$$
$$(x+5)(x-2)=0$$
$$x=-5$$
$$x=2$$

\section{Formal Logic}

A proposition is a statement that can be true or false.

\begin{equation}
	\bullet	
  \neg	
	\wedge	
	\vee	
	\oplus	
	\exists	
	\forall	
\end{equation}

\section{Fractions}

Multiplying fractions:
$$\frac{a}{b} * \frac{c}{d}$$
$$\frac{ac}{bd}$$

Dividing fractions:
$$\frac{a}{b} \div \frac{c}{d}$$
$$\frac{a}{b} * \frac{d}{c}$$
$$\frac{ad}{bc}$$

Dealing with subtracting fractions:
$$\frac{1}{a} - \frac{1}{b}$$
$$\frac{b}{ba} - \frac{a}{ba}$$
$$\frac{b-a}{ba}$$

Nested fractions:
$$\frac{\frac{1}{a} - \frac{1}{b}}{c}$$
$$\frac{b-a}{ab} * \frac{1}{c}$$
$$\frac{b-a}{abc}$$

Simplify nested fractions:
$$\frac{\frac{1}{2}}{\frac{3}{4}}$$
$$\frac{1}{2}*\frac{4}{3}$$
$$\frac{4}{6}$$

More Nested fractions:
$$\frac{\frac{1}{b}}{\frac{1}{b} - \frac{1}{a}}$$
$$\frac{\frac{1}{b}}{\frac{a-b}{ab}}$$
$$\frac{1}{b}*\frac{ab}{a-b}$$
$$\frac{ab}{b(a-b)}$$
$$\frac{a}{a-b}$$

Simple Nested division fractions:
$$\frac{a}{y} = \frac{a}{1}*\frac{1}{y}$$

More Nested fractions:
$$\frac{\frac{2}{y}+\frac{y}{2}}{y}$$
$$\frac{\frac{4+y^2}{2y}}{y}$$
$$\frac{4+y^2}{2y}*\frac{1}{y}$$
$$\frac{4+y^2}{2y^2}$$

Exponents of fractions:
$$\frac{7^{-3}}{7^{-1}} = \frac{1}{7^{2}}$$
$$\frac{7^{-3}}{7^{-1}} = 7^{-2}$$

Exponents More:
$$7*7*7*7*7 = \frac{7^{8}}{7^{3}}$$

Exponents With Easy:
$$(6^{3}*6^{-3})^{4}$$
Add when mult
$$(6^{0})^{4}$$
Mult when ``doubling''
$$6^{0}$$

\section{Expressions Structure}
$$a+b=-6$$
$$x+y+z=-2$$
$$8a-7x-7z-7y+8b$$
$$8(a+b)-7(x+y+z)$$
$$8(-6)-7(2)$$
$$1$$

\section{Fractional Exponents}
$$x^{1/3}=\sqrt[3]{x}$$
$$x^{1/5}=\sqrt[5]{x}$$

So: 
$$8^{1/3}=\sqrt[3]{8}$$
$$\sqrt[3]{8} = 2$$

\end{document}
