\documentclass{article}

\begin{document}

Hello world


\begin{equation}
k_{n+1} = n^2 + k_n^2 - k_{n-1}
\end{equation}

Khan Academy Question 2
Simplify the following for h
\begin{equation}
S=2\pi r(r+h)
\end{equation}
\begin{equation}
S/2\pi r=r+h
\end{equation}
\begin{equation}
h=(S/2\pi r)-r
\end{equation}

\begin{equation}
  S=250;
  r=4;
  h=(S/2\pi r)-r;
  h= 5.94718394325
\end{equation}


Does the following have an x-intercept?

\begin{equation}
  f(x) = x(x+4)
\end{equation}

% To graph, run calc. Type your range: e.g. _3 .. 3 then type the desired
% formula: x(x+4) and gf to run the graphing calc. It crosses the middle line, so
% I will guess yes, it has an x-intercept.

What is the rough y-intercept of $$[-3..3] f(x) = (x+3)(1-x) => ~2.8$$

The y-intercept is the value at f(0)
The x-intercept is the value where f(x) = 0
Periodic is a repeating or looping graph
An even graph has f(x) = f(-x) or will look the same reflected over the y-axis

What is the approximate maximum value of the formula:

\begin{equation}
  max(-x^2 + 6x - 1)  = ~8
\end{equation}


Solve for h
\begin{equation}
  A= 1/​2 ​​ (b+c)h
\end{equation}
\begin{equation}
  A/(1/2(b+c))= h
\end{equation}
\begin{equation}
  2A/(b+c)= h
\end{equation}

To store a variable and solve, use ``st'' and ``=''.

For example: I want to solve $$E/c^2$$
I type it exactly, then type my value for E, type stE, type my value for c, and type stc

What is the height of a trapezoid with one base equal to 20 m, the other base equal to 7m, and an area of 135m? It is 10m.


Solve $$m=E/(c^2)$$ for $$c = 300,000,000 m/s$$ and $$E = 1.8e14J$$

$$m = 2e-3$$

The Area of a trapezoid is $$A=(1/2)(b+c)h $$ solve for b
$$A/(1/2)(h)=b+c$$
$$(A/(1/2)h)-c=b$$
$$b = 2 A / h - c $$

When $$A=80ft^2$$ $$h=10ft$$ $$c=5ft$$

$$b=11$$


\end{document}
