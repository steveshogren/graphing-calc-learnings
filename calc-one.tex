\documentclass{article}

\begin{document}

Hello world

\begin{equation}
k_{n+1} = n^2 + k_n^2 - k_{n-1}
\end{equation}

Khan Academy Question 2
Simplify the following for h
\begin{equation}
S=2\pi r(r+h)
\end{equation}
\begin{equation}
S/2\pi r=r+h
\end{equation}
\begin{equation}
h=(S/2\pi r)-r
\end{equation}

\begin{equation}
  S=250;
  r=4;
  h=(S/2\pi r)-r;
  h= 5.94718394325
\end{equation}


Does the following have an x-intercept?

\begin{equation}
  f(x) = x(x+4)
\end{equation}

To graph, run calc. Type your range: e.g. [_3 .. 3] then type the desired
formula: 'x(x+4) and gf to run the graphing calc. It crosses the middle line, so
I will guess yes, it has an x-intercept.

What is the rough y-intercept of [-3..3] f(x) = (x+3)(1-x)? => ~2.8


\end{document}
